\documentclass[main.tex]{subfiles}
%%------ Preamble specific to this subfile only. This will be used when this document is compiled on its own. When the main document compiles this preabble is ignored. 
%%---------------------------------------------------------------------------%%
\begin{document}
\section*{Objectives}
\begin{comment}

    A well-written objective, often referred to as a learning objective or learning outcome, should be:
    1. **Specific:** It should clearly define what a student will be able to do by the end of the lesson.
    2. **Measurable:** It should be quantifiable in some way, whether that's through a test, an activity, or another form of assessment.
    3. **Achievable:** It should be realistic considering the students' current level of knowledge and the time allotted.
    4. **Relevant:** It should align with the overall goals of the course or curriculum.
    5. **Time-bound:** It should be achievable within a certain time frame, usually by the end of the lesson, unit, or course.
    This framework is often referred to as "SMART" objectives.
    The assessment should then be designed to directly measure whether or not each objective has been met. For instance, if your objective is "By the end of this lesson, students will be able to identify and explain the five major causes of World War I", your assessment might be a quiz that requires students to do exactly that. 

    Moreover, there should be a clear alignment (often referred to as "alignment of assessment") between your learning objectives and your assessment methods. For example, if one of your objectives requires critical thinking, then your assessment should not simply require factual recall. 

    Additionally, using a variety of assessment methods (such as quizzes, homework assignments, in-class discussions, and so on) can provide a more comprehensive picture of whether the students have met the objectives, as different students may excel in different types of assessments. 

    Ultimately, the objectives should guide the instruction and the assessment should measure the degree to which the instruction has enabled students to meet the objectives. This linkage between the objectives and assessment is critical to ensuring that the lesson is effective and that students' learning is maximized.

\end{comment}
\begin{enumerate}
  \item \textbf{Define and identify Variables:} Understand what a variable is and recognize its role in programming. Distinguish between different types of variables in VB.NET.

  \item \textbf{Understand and use Data Types:} Understand the concept of data types and how they structure the kind of data that can be stored. Be able to identify and utilize the basic data types in VB.NET including Integer, Double, String, and Boolean.

  \item \textbf{Assign values to Variables:} Know how to assign values to variables and modify these values as needed. Be able to use variables in simple mathematical operations.

  \item \textbf{Declare and use Constants:} Understand the concept of constants, know how to declare them, and understand why and when to use them.

  \item \textbf{Apply Variable Naming Conventions:} Understand and apply the naming conventions used for variables and constants in VB.NET.

  \item \textbf{Perform Type Conversion:} Understand the need for type conversions, and perform explicit and implicit type conversions in VB.NET.
\end{enumerate}

Self-Learning: Practice creating and manipulating variables and constants of different data types in a variety of contexts. The goal is to be able to comfortably create, assign, and manipulate variables and constants by the end of the session.
\end{document}

