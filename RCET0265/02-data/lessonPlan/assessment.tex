\documentclass[main.tex]{subfiles}

\begin{document}
\section*{Student Assessment} 
%%------ Preamble specific to this subfile only. This will be used when this document is compiled on its own. When the main document compiles this preabble is ignored. 
%%---------------------------------------------------------------------------%%
% In the context of a lesson plan, the "Assessment" section refers to how the instructor plans to evaluate or measure students' understanding and learning of the lesson's objectives.
% Assessment can take various forms, including but not limited to:
% 1. Quizzes and Tests: These can be given during or after the lesson to evaluate students' knowledge and understanding.
% 2. Classroom Activities: Such as group work, presentations, and in-class assignments. These can provide an opportunity to assess both individual and group understanding.
% 3. Homework Assignments: Work assigned to be completed outside of class can provide another way to assess student understanding.
% 4. Class Participation: The level and quality of student involvement in class discussions can also be a part of assessment.
% 5. Observation: Teachers can informally assess student understanding by observing their engagement, asking questions, and encouraging classroom discussion.
% In your lesson plan, under the "Assessment" section, you would list the methods you intend to use to gauge students' understanding of the lesson's material. For example, you might plan to give a short quiz at the end of the lesson, assign homework, or observe student participation during a class discussion.
Potential assessments for each of the learning objectives:
\begin{enumerate}
  \item \textbf{Define and identify Variables:} Provide a piece of code with several variables and ask students to identify them. Ask them to explain in their own words what variables are and how they function within the program.

  \item \textbf{Understand and use Data Types:} Ask students to write a short piece of code that declares variables of each data type (Integer, Double, String, Boolean) and assigns them appropriate values. They should also provide a brief explanation of why they chose those specific values for each data type.

  \item \textbf{Assign values to Variables:} Provide a piece of code where several variables need to be assigned values. Students should assign values to those variables and explain why they chose the values they did.

  \item \textbf{Declare and use Constants:} Ask students to write a piece of code where they declare a constant and use it in a meaningful way. They should also explain why a constant was appropriate for this use case.

  \item \textbf{Apply Variable Naming Conventions:} Provide a piece of code with improperly named variables and ask students to correct them according to VB.NET naming conventions.

  \item \textbf{Perform Type Conversion:} Ask students to write a program where they need to convert between different data types. They should explain why the type conversion was necessary in each case.

\end{enumerate}

Remember, for each assessment, it's important to evaluate not only if the student completed the task, but also if they understood why they did what they did. Always encourage them to explain their reasoning.

\end{document}
