\documentclass[main.tex]{subfiles}
%%----- Preamble specific to this subfile only. This will be used when this document is compiled on its own. When the main document compiles this preabble is ignored. 
\usepackage{listings}
%%--------------------------------------------------------------------------%
\begin{document}
\section*{Lesson Plan}

%%\item-----   New command for lesson plan item entries
\begin{comment}
  %Example:
  
  \planItem{5}                %time in minutes
  {Introduction}              % title/name of lesson plan item
  {lesson notes and content}  % this can be a paragraph or a list
  {url}                       % link(s) to resources

\end{comment}

\planItem{10}
{Define and identify Variables}
{ \begin{itemize}
  \item Introduction to the concept of variables.
  \item Discussion on the purpose and use of variables in programming.
  \item Demonstration: How to declare and use variables in VB.NET.
  \item Practice Activity: Students will write code to declare and use variables.
  \item Assessment: Students will identify variables in given code snippets and explain their purpose.
\end{itemize}
}
{}

\planItem{10}
{Understand and use Data Types}
{ \begin{itemize}
  \item Introduction to the concept of data types.
  \item Explanation of the main data types in VB.NET - Integer, Double, String, Boolean.
  \item Demonstration: How to declare variables of different data types and assign them values.
  \item Practice Activity: Students will write code using different data types.
  \item Assessment: Students will write code that declares and uses variables of each data type, explaining their choices.
\end{itemize}
}
{}

\planItem{10}
{Assign values to Variables}
{ \begin{itemize}
  \item Discussion on assigning and changing variable values.
  \item Demonstration: Assigning values to variables and using them in simple mathematical operations.
  \item Practice Activity: Students will write code to assign values to variables and modify them.
  \item Assessment: Students will be given code where they will assign values to variables and explain their choices.
\end{itemize}
}
{}

\planItem{10}
{Declare and use Constants}
{ \begin{itemize}
  \item Introduction to the concept of constants and when to use them.
  \item Demonstration: Declaring and using constants in VB.NET.
  \item Practice Activity: Students will write code using constants.
  \item Assessment: Students will write code to declare and use a constant, explaining their choice of when to use the constant.
\end{itemize}
}
{}

\planItem{10}
{Apply Variable Naming Conventions}
{ \begin{itemize}
  \item Discussion on VB.NET naming conventions for variables and constants.
  \item Demonstration: Correct and incorrect examples of variable and constant names.
  \item Practice Activity: Students will write code applying correct naming conventions.
  \item Assessment: Students will correct improperly named variables in given code, conforming to VB.NET naming conventions.
\end{itemize}
}
{}

\planItem{10}
{Perform Type Conversion}
{ \begin{itemize}
  \item Explanation of the need for type conversions.
  \item Demonstration: Performing explicit and implicit type conversions in VB.NET.
  \item Option Strict On
  \item Demonstration: Performing explicit and implicit type conversions with Option Strict On
  \item Practice Activity: Students will write code to convert between different data types.
  \item Assessment: Students will write a program that involves type conversions and explain why each was necessary. 
\end{itemize}
}
{}
%
\planItem{5}
{test}
{
% {  \begin{figure}[h!]
    % \centering
    % \begin{lstlisting}[language=VBScript]
  % \verb+Dim doubleNumber As Double = 10.5 \\
  % \verb+Dim integerNumber As Integer \\
  % \verb+integerNumber = CType(doubleNumber, Integer) `Explicit conversion from Double to Integer, fractional part will be truncated  \\
    % \end{lstlisting}
  % \end{figure}
  % \texttt{ 
  %         Dim integerNumber As Integer = 10 \\
  %         Dim doubleNumber As Double \\
  %         doubleNumber = integerNumber  'Implicit conversion from Integer to Double 
  %         }
}
{}
% \planItem{5}
% {Perform Type Conversion}
% { 
% \begin{itemize}
%   \item Implicit Conversion: \\
%   \texttt{ 
%           Dim integerNumber As Integer = 10 \\
%           Dim doubleNumber As Double \\
%           doubleNumber = integerNumber  'Implicit conversion from Integer to Double \\
%         }
%   \item Explicit Conversion:
% \end{itemize}
%     \begin{lstlisting}[language=VBScript]
%     %         Dim doubleNumber As Double = 10.5
%     %         Dim integerNumber As Integer
%     %         integerNumber = CType(doubleNumber, Integer)  'Explicit conversion from Double to Integer, fractional part will be truncated
%     \end{lstlisting}
% }  
% {}
%
\begin{comment}
Sure, here are the lesson plan sections for each of the learning objectives you mentioned:

For each section, the lesson will start with a brief introduction, followed by a demonstration, then a practice activity, and will conclude with an assessment to ensure understanding. Feedback will be provided throughout to ensure students are grasping the concepts.
Absolutely, here are code examples for each of the learning objectives:

1\planItem{5}
{Define and identify Variables}
{ \begin{itemize}
   \item Declaration: `Dim name As String`
   \item Use: `name = "John Doe"`

2\planItem{5}
{Understand and use Data Types}
{ \begin{itemize}
   \item Integer: `Dim age As Integer = 30`
   \item Double: `Dim average As Double = 85.6`
   \item String: `Dim greeting As String = "Hello"`
   \item Boolean: `Dim isReady As Boolean = True`

3\planItem{5}
{Assign values to Variables}
{ \begin{itemize}
   \item `Dim count As Integer`
   \item `count = 10`  'Assigning a value to the variable
   \item `count = count + 1`  'Changing the value of the variable

4\planItem{5}
{Declare and use Constants}
{ \begin{itemize}
   \item Declaration and use: 
     ```vbnet
     Const PI As Double = 3.14159
     Dim radius As Double = 2.5
     Dim area As Double = PI * radius * radius
     ```

5\planItem{5}
{Apply Variable Naming Conventions}
{ \begin{itemize}
   \item Good practice: `Dim firstName As String`
   \item Bad practice: `Dim firstname As String` or `Dim first_name As String`

\planItem{5}
{Perform Type Conversion}
{ \begin{itemize}
   \item Implicit Conversion:
     ```vbnet
     Dim integerNumber As Integer = 10
     Dim doubleNumber As Double
     doubleNumber = integerNumber  'Implicit conversion from Integer to Double
     ```
   \item Explicit Conversion:
     ```vbnet
     Dim doubleNumber As Double = 10.5
     Dim integerNumber As Integer
     integerNumber = CType(doubleNumber, Integer)  'Explicit conversion from Double to Integer, fractional part will be truncated
     ```
   
These examples illustrate the basic usage of variables, data types, constants, naming conventions, and type conversions in VB.NET. You can use these examples as a starting point and modify them to fit your specific lesson plan needs.
\end{comment}
\end{document}
